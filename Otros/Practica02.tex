\documentclass{article}
\usepackage{graphicx} %pa' las imágenes
\usepackage[utf8]{inputenc} %pa' los acentitos
\graphicspath{{/home/Valeria/Practica02/Otros}}

\title{\Huge Taller de Herramientas Computacionales}
\author{\huge Valeria Ortiz Cervantes}
\date{13 de marzo del 2019}

\begin{document}
\maketitle
\begin{center}
	\subsection*{\LARGE Universidad Nacional Autónoma de México.\\Facultad de Ciencias.\\}
	\includegraphics[scale=3]{/1.jpg}
\end{center}
\newpage
\begin{center}
	\title{\LARGE Práctica 1.Preguntas}
\end{center}
\section*{Algunos comandos de la terminal.}
\begin{enumerate}
	\item mkdir :para crear una carpeta nueva.
	\item sudo su : para acceder con derechos de administrador.
	\item cd : para cambiar de directorio, se utiliza junto al nombre del directorio.
	\item ls : muestra el contenido de un directorio, también se debe poner el nombre del directorio al lado. 
	\item bg : pasa a ejecutar un programa en segundo plano.
	\item fg : regresa un programa a primer plano.
	\item rmdir : para borrar directorios vacíos.
	\item clear : limpia la pantalla de la terminal. 
	\item date : indica la fecha y hora.
	\item cal : muestra un calendario del mes actual.
\end{enumerate}
\section*{Características mínimas de un lenguaje.}
La mayoría de los lenguajes de programación tienen los siguientes elementos en común:
\begin{enumerate}
	\item Caracteres: el tipo de caracteres que acepta el lenguaje. 
	\item Tipos de datos: la mayoría de los programas operan con datos, por lo cual en el código fuente debemos definir qué tipo de datos vamos a utilizar en el programa. 
	\item Instrucciones: ayudan a manipular los datos que hemos definido.
	\item Comentarios: para documentar el programa, sirve mucho a la hora de revisarlo de nuevo mucho tiempo después.
	\item Identificadores: es el nombre de cualquier variable, constante, procedimiento o programa.
	\item Expresiones y operadores: Una expresión es una combinación de operadores y operandos de cuya evaluación se obtiene un valor. Los operandos pueden ser nombres que denoten objetos: variables o constantes, funciones, etc. 
\end{enumerate}
\section*{Instrucciones "for" y "while".}
¿Por qué si las instrucciones for y while son equivalentes existen en todos los lenguajes de programación?\\Ambas funciones tienen la misma base, ejecutan una instrucción o un bloque de instrucciones repetidamente hasta que una determinada expresión se evalúa como falsa. \\La diferencia entre for y while es que for es más útil para hacer iteraciones, como recorrer los elementos de una lista; mientras que while se ejecuta mientras una condición en la sentencia permanezca como verdadera.
\end{document}